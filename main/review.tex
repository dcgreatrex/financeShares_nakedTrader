%------------------------------------------------------------------------------------------------
% Title:             Review of Burns, R (2011). The Naked Trader: How anyone can make money trading shares
% Author:         D.C.Greatrex
% Date:            28/06/2016
% Version:	1.0
% Changes:	 -     
%------------------------------------------------------------------------------------------------
\section*{Introduction - Day in the life of the Naked Trader}
\begin{itemize}
\item 07:00: Check whether there are any announcements on the shares that he owns. 
Judge whether any action is required. 
For example, should you sell and take profits? 
Perhaps, buy some more.
Also check the spread betting websites to see how the FTSE 100 is set to open.
\item 08:00: Check all positions to see if any positions have crashed for any reason.
It is usually not worth trading in the first hour of the day.
This is because there is always an over reaction to news which can be magnified by small volume and large spreads.
\item 09:45: Check the market. It is around this time that he considers a trade.
He never rushes anything and usually calls a trading friend to discuss ideas and options.
\item 10:00: Read the daily news paper.
Review the current portfolio - is there anything bothering him? 
Look at level 2 data to get a feel for liquidity.
He only buys if the timing is right.
The worst thing to do is to make a boredom trade!
He continues researching any news articles that he noticed earlier in the day.
\item 11:00: The markets go quiet around 11am.
\item 11:45: Look through short list of potential buy and sell shares.
Have any been rising or falling significantly in the morning?
This is probably the best time of the day to buy.
\item 13:30: The market sometimes changes around 13:30 due to the US and can and shares can often move.
This does not both him too much, however if he is betting on the FTSE falling via spread betting he will usually take profits at this point.
\item 14:15: Takes a look at his higher risk short list of shares.
\item 14:30: The Dow Jones opens,
UK shares tend to follow the US movements.
He will use this time to decide whether to sell falling shares or to buy shares just after a negative overreaction in the US.
\item 15:00: Have a chat with his trading friends to share observations and ideas with one another.
\item 16:00: Take one last look at the portfolio.
How did it do?
What did he miss?
How much money was made or lost?
\item 16:30: Take one last look though ADVFN.com then turn off the computer and spend time with his family.
\item 21:00: Check to see where the Dow Jones finished the day.
The US market will usually have a strong impact on where his portfolio will open in the morning.
Spend a little bit of time researching companies flagged up by the mornings news feed.
Check to see whats being launched on the market over the coming days.
\item 22:00: Have one last check of the news.
\end{itemize}

\section*{Warning}
Do not share trade if you 1) cannot afford to loose a large percentage of your starting capital and 2) ARE A COMPULSIVE GAMBLER.

%------------------------
\section*{Section 1 - Getting started}


%------------------------
\subsection*{Starting capital}
If you only have \pounds 500 there is little point share trading as commission costs will significantly eat into any potential gains.
In this scenario wait until you have more capital and paper trade in the mean time.
You should have around \pounds 3000 or more before you start to buy shares.
To become a full time trader you will realistically need a starting capital of 4 * your required income.
In other words, you can and need to be looking to make between 20-25\% annual profit on the investment capital.
\subsection*{Types of traders - See the problems so you know what to avoid}
\begin{enumerate*}
\item Small cap oil/miner: Over exposed to small cap oil miners in the hope of finding a massive riser.
Too often invests in bad shares and ends up loosing large amounts of capital. 
\item Day trader: 95\% of day traders lose.
It is very hard to do well and along with high stress levels, most people are unable to make a living from it.
\item Chart guy: Concentrates purely on technical analysis. 
In the naked traders opinion these traders are missing the bigger picture.
Charts should be used along side other research methods.
\item System addict: Buys expensive black box trading systems and is unwilling to carry out proper research.
Mr Safe and Steady: Only buys shares that are perceived as being 'safe'.
\item Long term investor: Does adequate research but holds onto shares for too long before taking profit.
\item Medium term investor: Buys and holds shares for between 3 months and 2 years. 
Does careful research, sticks to stop losses, takes profit when necessary and keeps an eye on the portfolio.
\item Analyser: Over-analyses and avoids buying shares due to worry.
\item Accountant: Only focuses on fundamental ratio and as a result misses out on a lot of opportunities.
\item Shorter: Only focuses on shorting where as over time, shares go up more than they go down.
Bull markets last longer than bear markets.
\item In and out: Often buys looking for a quick rebound, however sometimes this does not happen which reduces profits.
\item Bottom picker: Only buys failing stocks looking for overselling and cheap shares.
Sometimes however bottomed shares do not rise or can fall further.
\item Scaredy cat: Sells the moment the market goes down and buy in again at a higher price.
\item Bulletin board: Spends too much time on bulletin boards.
Do your own research and do not trade on others tips!
\item Penny share: Only buys shares cheaper than 20p in the hope of finding large risers.
These shares are more likely to go bust than make the FTSE main market.
\item Indices: Does not bother with shares an only bets on indices.
Most people lose playing indices.
\end{enumerate*}

Buy knowing about these trading styles you will be able to identify when you fall into them and then correct your behaviour.

%------------------------
\subsection*{Toolkit}
Avoid capital gains tax by trading via a stocks and shares isa. This requires setting up an account with a UK broker.
Do not bother with an advisory broker just use an execution broker and do your own research.

Use real time prices by signing up to an online website such as ADVFN, moneyAM, Proquote, Hemscott, Digitallook.
When you see the offer (buy), bid (sell) and current price, realise that due to the spread you are always losing money as soon as you buy any share.

Set up a watch list of a range of buy and sell potential shares.
Use a note book to document your investment ideas. 
It is also good practice to keep a trading diary of every historic and current trade that you make.
This is useful when analysing your performance to understand why you made or lost money.

Currently on page 55.


%------------------------------------------------
\section*{Section 2}





%\begin{align}
%A = 
%\begin{bmatrix}
%A_{11} & A_{21} \\
%A_{21} & A_{22}
%\end{bmatrix}
%\end{align}

%\begin{itemize}
%\item First item in a list 
%\item Second item in a list 
%\item Third item in a list
%\end{itemize}

%\begin{table}
%\caption{Random table}
%\centering
%\begin{tabular}{llr}
%\toprule
%\multicolumn{2}{c}{Name} \\
%\cmidrule(r){1-2}
%First name & Last Name & Grade \\
%\midrule
%John & Doe & $7.5$ \\
%Richard & Miles & $2$ \\
%\bottomrule
%\end{tabular}
%\end{table}

%\begin{description}
%\item[First] This is the first item
%\item[Last] This is the last item
%\end{description}
%------------------------------------------------------------------------------------------------