%------------------------------------------------------------------------------------------------
% Title:             Review of Burns, R (2011). The Naked Trader: How anyone can make money trading shares
% Author:         D.C.Greatrex
% Date:            28/06/2016
% Version:	1.0
% Changes:	 -     
%------------------------------------------------------------------------------------------------
%\section*{Introduction - Day in the life of the Naked Trader}
%\begin{itemize}
%\item 07:00: Check whether there are any announcements on the shares that he owns. 
%Judge whether any action is required. 
%For example, should you sell and take profits? 
%Perhaps, buy some more.
%Also check the spread betting websites to see how the FTSE 100 is set to open.
%\item 08:00: Check all positions to see if any positions have crashed for any reason.
%It is usually not worth trading in the first hour of the day.
%This is because there is always an over reaction to news which can be magnified by small volume and large spreads.
%\item 09:45: Check the market. It is around this time that he considers a trade.
%He never rushes anything and usually calls a trading friend to discuss ideas and options.
%\item 10:00: Read the daily news paper.
%Review the current portfolio - is there anything bothering him? 
%Look at level 2 data to get a feel for liquidity.
%He only buys if the timing is right.
%The worst thing to do is to make a boredom trade!
%He continues researching any news articles that he noticed earlier in the day.
%\item 11:00: The markets go quiet around 11am.
%\item 11:45: Look through short list of potential buy and sell shares.
%Have any been rising or falling significantly in the morning?
%This is probably the best time of the day to buy.
%\item 13:30: The market sometimes changes around 13:30 due to the US and can and shares can often move.
%This does not both him too much, however if he is betting on the FTSE falling via spread betting he will usually take profits at this point.
%\item 14:15: Takes a look at his higher risk short list of shares.
%\item 14:30: The Dow Jones opens,
%UK shares tend to follow the US movements.
%He will use this time to decide whether to sell falling shares or to buy shares just after a negative overreaction in the US.
%\item 15:00: Have a chat with his trading friends to share observations and ideas with one another.
%\item 16:00: Take one last look at the portfolio.
%How did it do?
%What did he miss?
%How much money was made or lost?
%\item 16:30: Take one last look though ADVFN.com then turn off the computer and spend time with his family.
%\item 21:00: Check to see where the Dow Jones finished the day.
%The US market will usually have a strong impact on where his portfolio will open in the morning.
%Spend a little bit of time researching companies flagged up by the mornings news feed.
%Check to see whats being launched on the market over the coming days.
%\item 22:00: Have one last check of the news.
%\end{itemize}

%------------------------------------------------
\section*{Warning}
Do not share trade if you 1) cannot afford to loose a large percentage of your starting capital and 2) ARE A COMPULSIVE GAMBLER.

%------------------------------------------------
\section*{Section 1 - Getting started}

%------------------------
\subsection*{Starting capital}
If you only have \pounds 500 there is little point share trading as commission costs will significantly eat into any potential gains.
In this scenario wait until you have more capital and paper trade in the mean time.
You should have around \pounds 3000 or more before you start to buy shares.
To become a full time trader you will realistically need a starting capital of 4 * your required income.
In other words, you can and need to be looking to make between 20-25\% annual profit on the investment capital.

%------------------------
\subsection*{Types of traders: Know what to avoid}
\begin{enumerate*}
\item Small cap oil/miner: Over exposed to small cap oil miners in the hope of finding a massive riser.
Too often invests in bad shares and ends up loosing large amounts of capital. 
\item Day trader: 95\% of day traders lose.
It is very hard to do well and along with high stress levels, most people are unable to make a living from it.
\item Chart guy: Concentrates purely on technical analysis. 
In the naked traders opinion these traders are missing the bigger picture.
Charts should be used along side other research methods.
\item System addict: Buys expensive black box trading systems and is unwilling to carry out proper research.
Mr Safe and Steady: Only buys shares that are perceived as being 'safe'.
\item Long term investor: Does adequate research but holds onto shares for too long before taking profit.
\item Medium term investor: Buys and holds shares for between 3 months and 2 years. 
Does careful research, sticks to stop losses, takes profit when necessary and keeps an eye on the portfolio.
\item Analyser: Over-analyses and avoids buying shares due to worry.
\item Accountant: Only focuses on fundamental ratio and as a result misses out on a lot of opportunities.
\item Shorter: Only focuses on shorting where as over time, shares go up more than they go down.
Bull markets last longer than bear markets.
\item In and out: Often buys looking for a quick rebound, however sometimes this does not happen which reduces profits.
\item Bottom picker: Only buys failing stocks looking for overselling and cheap shares.
Sometimes however bottomed shares do not rise or can fall further.
\item Scaredy cat: Sells the moment the market goes down and buy in again at a higher price.
\item Bulletin board: Spends too much time on bulletin boards.
Do your own research and do not trade on others tips!
\item Penny share: Only buys shares cheaper than 20p in the hope of finding large risers.
These shares are more likely to go bust than make the FTSE main market.
\item Indices: Does not bother with shares an only bets on indices.
Most people lose playing indices.
\end{enumerate*}

Buy knowing about these trading styles you will be able to identify when you fall into them and then correct your behaviour.

%------------------------
\subsection*{Toolkit}
Avoid capital gains tax by trading via a stocks and shares isa. This requires setting up an account with a UK broker.
Do not bother with an advisory broker just use an execution broker and do your own research.

Use real time prices by signing up to an online website such as ADVFN, moneyAM, Proquote, Hemscott, Digitallook.
When you see the offer (buy), bid (sell) and current price, realise that due to the spread you are always losing money as soon as you buy any share.

Set up a watch list of a range of buy and sell potential shares.
Use a note book to document your investment ideas. 
It is also good practice to keep a trading diary of every historic and current trade that you make.
This is useful when analysing your performance to understand why you made or lost money.
It is also important that your trading environment is clutter free and that you are in an environment in which you remain clear headed.

Finally, Keep It Simple Stupid (KISS). 
You do not need to buy complicated chart packages or trading systems to trade shares.
All you need is access to news stories, fundamentals, charts and real time prices.

%------------------------
\subsection*{Basic knowledge}
To begin trading you need to understand the following topics:
\begin{enumerate*}
\item Spreads: As soon as you buy a share you are down until the bid price rises above the offer price.
Spread size is negatively correlated with market/company size.
A spread of 4\% is the maximum the naked trade will allow when buying.
\item Trading costs: Commission costs add up. Over trade at your peril!
\item Exchange market size (EMS): EMS are the amount of shares market makers guarantee to sell or buy at quoted prices.
Each share has a quoted EMS which is found on the London Stock Exchange website. 
If you own more shares than then quoted EMS you may be charged extra to sell them in a market panic.
Don't deal in shares with a EMS equivalent to less than \pounds 2000.
\item Trading hours:
\item Why shares move up and down: 'Buy on rumour, sell on news'.
Broker upgrade/downgrade. 
Market/sector move.
Institution move - Look for RNS entitled 'Holdings in Company'.
Director buying and selling. News. Dividend dates. Magazine tips. Bulletin board manipulation. Market maker manipulation. Treeshake. Surprise event. 
Rights issue - Company raises money buy offering shares at a lower price. This usually drops the share price.
Takeover/merger - Usually increases share price. 
\item Market makers and crowds: Market makers ensure that small company shares always have liquidity.
The more traders buy and sell, the more money market makers make. 
It is therefore in their interests to move the market around to encourage buying and selling.
\item Treeshaking: When market makers drop the share price for no reason.
They are designed to make you afraid.
If market makers have a big buy order they will drop the price to encourage others to sell which they can then buy immediately for their clients.
Do not panic in a treeshake. Calmly check the news for legitimate market moving information.
If no news, go and make a cup of tea and relax.
Treeshakes usually last a few hours buy which time the share price will have recovered.
If it persists over a few days there could be company information that you are unaware of.
\item Dividends: Dividends add up and are a good way of covering trading costs.
Be aware of the ex-dividend date for each stock that you own or are interested in.
\end{enumerate*}

%------------------------
\subsection*{Losses}
Taking a lot of small loses will make you a lot of money.
Get rid of losses when they are little and keep onto profits when they grow - Always sell losses early!
Take no more than 10\% loss on a trade and look for 30\% + on really good risers.
Emotions lead to traders taking profit too quickly which harms long term investment gains.
The inability to sell at a loss however is due to ego.
Always buy winners and sell losers, not the other way around.

Gambling/Investing: The majority of spread betting accounts are owned by people who gamble on the market without doing research.
Don't have 'punts', invest.
'Punts' will lose eventually.

%------------------------
\subsection*{General rules}
\begin{enumerate*}
\item Don't over trade.
\item Buy shares trading upwards and breaking new highs. Don't buy shares trending downwards and making new lows.
\item Learn to love buying shares in an up-trend.
\item Adding as the up-trend continues is the way to make money.
\item You have to stay cool when the markets are not doing as you expect.
\item Don't try to make trading decisions when you are tired, hungover or ill.
\end{enumerate*}

%------------------------------------------------
\section*{Section 2 - Picking shares}
The Naked Trader looks for trading ideas in: Newspaper round-ups, investment publications (Investors Chronicle, Shares magazine, International business times research), ADVFN newswire, ADVFN Top lists, percentage gainers and losers, ADVFN Breakouts. 
Some of these services are free, others require premium membership to the site.
The Naked Trader tends to favour shares that have a market cap of between \pounds 50 million and \pounds 900 million as they have a lot more room to grow compared with FTSE 100 companies.
These can be found in the FTSE 250 and FTSE small cap indices. 

%------------------------
\subsection*{Selection methodology}
Do not skip the research when picking a share. 
First build up the story of the share and then decide if you want to buy the story.
Uncover the following information:
\begin{enumerate*}
\item How much is the company worth?
\item Full year pre-tax profits?
\item Are dividends rising?
\item Is the outlook positive and are there negative things happening?
\item Net debt, size of dividends, purpose of company, sector, state of the current market and sector.
\item When are next statements due and is the share up-trending?
\end{enumerate*}
When searching for information do not trust figures published on trading sites, rather lift them straight from the most recent company reports.

\paragraph{The 'secret' traffic light system:} Go on ADVFN, click on the news tab and then on 'Highlight Phrases'.
Add the following words to the Highlight Phrases filter with corresponding colours:
\begin{enumerate*}
\item RED filter: [Challenging, difficult, down by, unpredictable, lower, poor, difficult trading, tough, below expectations, deficit].
\item YELLOW filter: [In line with expectations, cash].
\item GREEN filter: [Exceeding expectations, positive, favourable, profit up, excellent].
\item BLUE filter: [Debt, covenants, borrowings].
\end{enumerate*}
Now use the phrase filters to look through news stories and reports of companies you are interested in.
Lots of reds = Sell or not a buy. 
Lots of yellow = hold.
Lots of green = potential buy.
Lots of blue = debt heavy.
Do not buy companies with net debt more than three times the full-year pre-tax profits.

\paragraph{Additional checks}
Once you have a short list ensure that dividends are rising, the chart setup looks good and that the company has a clear news history.
Next check the director dealings and the company website to get a feel for how they sell themselves and value shareholders.
When viewing director dealing always ask how many shares did they buy or sell RELATIVE to the amount they currently own.
Also check the Price to Earnings P/E ratio.
P/E represents the number of years it will take for the earnings of the company to cover the share price.
Remember that P/E are sector specific and only good for comparing cross company. 
The Naked Trader prefers to focus on companies with lower P/E in the range of 12-20 as they are likely to have greater growth potential.
Finally, when picking undervalued shares the Naked Trader will only consider companies with a market cap of up to 15 * profits.

%------------------------------------------------
\section*{Section 3 - "Winning" trading strategies}
Burns presents twenty tried and tested trading strategies in the book.
Each strategy is intended only for well researched securities that fit the selection criteria outlined in the rest of this article:
\begin{itemize}
\item FTSE 100 promotion:
Buy shares just before they get into the FTSE 100 from the FTSE 250.
Check for shares with a market-cap up-trending towards �3bn (around �2.7-3bn).
Focus on high ranked shares in the FTSE 250.
Take profits a few days after promotion.
\item "Ahead of Expectations":
Look for companies that publish RNS feeds with the phrase "above expectations" or similar.
Best time/place to find these statements is when share related news breaks at 7am.
Keep filtering for positive news.
\item Retailer recommendation:
Buy/sell shares in retailers that your partner and friends rate/hate and believe are in fashion/out of fashion.
In particular ask friends who are rich or have large disposable incomes.
Don't hold onto retailer shares for too long as fashion and trends change relatively quickly. 
\item Game changers:
Look for companies that have recently uncovered a new/innovative revenue stream or disruptive technology.
Similarly, those that are cutting loss making divisions whilst simultaneously investing in new innovative/original projects.
Look for the word 'transformational' in the company reports.
\item Find something cheap:
Look for market cap that is low compared with full year profits or forecast future full year profits.
Ensure that debt is not an issue with a candidate company.
\item New production facilities in industrials:
Check company reports for mention of any new build production facility.
This is likely evidence of a large order book and high growth potential for the coming year.
\item Buy shares in a 'hot' sector:
Use common sense and indications of general share price rises when looking for 'hot' sectors.
A 'hot sector' will not remain hot forever, but ride the wave until there are considerable signs of an obvious downturn.
If you find a share you like, check other shares in it's sector to ensure movements are not share but sector specific.
Do your normal research when selecting specific shares, however do not be afraid to buy strongly into hot sectors.
\item Recovery shares:
Look for shares that have decreased a lot and are begin to up-trend again.
You need to have a clear and tangible reason for why the share will recover before buying - likely news based.
Never buy the share if its price is still decreasing and check debt figures to ensure that the company will not go bust.
\item Oscillatory trends:
If a share price is in an oscillatory trending pattern over a sustained period buy at the lows and short at the highs for lots of short term gains.
Sell if the trending range is broken either way and use stop losses $\pm$2\% either side of the range.
\item Buy boring companies for defence and dividends.
\item Buy just before share splits, consolidations and bonus issues:
Find out about share splits in early morning company reports.
\item Strong companies in niche markets:
These companies are more likely to be bought out and will have strong profit margins.
Buy and hold whilst the market grows.
\item Bid targets:
Companies which are bid for by other entities usually experience a large increase in share price.
Use common sense when indentifying these companies by thinking from a buyers perspective and reading related news.
Also check for sudden large volume buying activity.
Institutions tend to buy in around 6 weeks before bids in smaller companies due to insider trading. 
\item Shares moving from AIM to main market:
Buy shares moving onto the main market as they are growth companies which tracker funds will then be able to buy.
Look in company reports for notification of their intent to move markets.
\item Out-performing company division: 
Look for companies who are profitable but currently have an outperforming sub-division.
Company revenue will ensure that the share is secure whilst activity from the performing division drives up the share price.
\item Small oil exploration companies:
High risk.
Look for a strong management and only allocate a small percentage of capital to this subsector.
Has it found oil? Is it funded? What are the management like? Is there scope for growth? 
\item Metals: 
Metal companies are hard to value therefore rely on the underlying market momentum to make trading decisions.
Allocate a small percentage of capital to metal producers and use a trailing stop loss to take profits when any significant pull backs occur.
\item The future:
Identify companies whose product and sector will likely have high relevance in the future environment.
Once opportunities are identified, hold for the long term.
\item Long term profit and dividend growth:
Find companies who have raised profits and dividends for at least three years and who have net cash and low debt.
Hold the share until doubts are raised concerning performance or growth.
\item Buy what you know:
Use industry and company specific knowledge to identify opportunities and risks in a share.
\item Buy new IPO listings:
Look for companies valued over \pounds 400 million as they will get automatic entry into either the FTSE 100 or 250 and will be picked up by tracker funds.
Avoid over valuations that do not match the fundamentals. 
Oil and energy IPOs have a strong likelihood of producing gains.
To buy in, call your broker on the morning of the IPO saying that you want a quote.
\item Times of the year: 
There are historically good and bad days and months with which to trade the London stock exchange.
This is probably down to psychological factors as well as holiday periods which cause low liquidity.
\begin{itemize}
\item Good times to buy: 
April.
August - watch out for low liquidity and higher volatility though. 
Also look out for good company reports issued in this month.
October \& November are usually quite calm but if you find the right shares they can increase in profit in the run up to Christmas.
December and January are hot months.
There is a 69\% probability (at the time of publication) of FTSE positive returns in this month.
There is usually a buying squeeze between Christmas and New Year due to low liquidity which can move share prices.
Ensure you use stop losses and watch out for bad company news being published in holiday periods.
\item Bad times to buy: 
February, March, May - July \& September.
\end{itemize}
\item Market downturn and turmoil:
\begin{itemize}
\item Understand what type of market downturn you are in.
Is it caused by a short technical / global event? 
Is it the start of a global recession and the end of the monetary system as we know it?
\item Avoid selling all of your position as commission costs add up - especially when having to buy back in at a later date.
Instead hedge your investments by buying a leveraged short etf of the underlying market that you are invested in.
Increase or decrease your hedged position depending on which way the market turns.
\item Ensure that you have cash in your account for buying in at the turn.
\item Consider buying at the dips of large 'black swan' (improbably) global events as strong companies will usually bounce back quickly.
\item Buy when everyone is selling. Sell when everyone is buying.
\item Cut stakes in volatile stocks.
\end{itemize}
\end{itemize}

%------------------------------------------------
\section*{Section 4 - Charts, timing, targets and stops}
The Naked Trader warns against over reliance on charts without accounting for company fundamentals. 
Regardless, he describes a number of ways in which he uses charts to aid his trading.
They are:
\begin{itemize}
\item Identify a price range or channel in which a share is trading.
\item Identify a break out from a previously established price range.
\item Identifying resistance and support levels for the purpose of setting stop losses and take profit targets.
Stop losses should be set slightly below lower support levels.
Lower and upper support levels can function as trading signals.
\item Identify upwards trending stocks and channels.
\end{itemize}

\paragraph{Charting patterns:}
The following chart patterns are recommended for further study, even though the Naked Trade remains cynical about their reliability:
\begin{enumerate*}
\item Buy signals patterns: "The Double Bottom", "The Cup and Handle", "Round Bottom".
\item Sell signal patterns: ""Head and Shoulders", "The Double Top", "The Triple Top".
\end{enumerate*}
The Naked trader recommends "The Investor's Guild to Charting" by Alistair Blair as a resource for investors interested in pattern strategies.  

\paragraph{Market timing:}
"Timing is everything...with shares".
The Naked Trader always questions whether the day and time is the best possible entry point for buying into a company on his buy list.
This requires patience.
Once you have identified a share, add it to your watch list and monitor its progress.
Before buying you should check for:
\begin{enumerate*}
\item High volume.
\item Positive momentum.
\item Positive chart.
\item Date of next statement.
\item Market and sector conditions.
\item Level 2.
\item Usual movement patterns for the share.
\end{enumerate*}
The Naked Trader only buys in on a day in which the share price is rising.

Planning your trades is very important and timing must play a large role in these decisions.
You should have an entry price, time, take profit price and stop loss levels before entering into the market.
Around four to six weeks before company results is often a good time to buy due to the possible positive momentum leading up to the publication.

\paragraph{Volume:}
High volume usually means something interesting is going on.
Understand what the average trading volume is for a share by viewing all previous trades over the past four weeks and then compare this with the volume over the last few days.
When looking at trades reported online, remember that their classification is the best guess of the computer and a trade marked as "buy" could well be a "sell".
Therefore treat trading volume with caution and assume that if a share is going up it is likely to have more buyers and vis-versa.
There are different types of trades and are reported by the LSE with a code describing the transaction:
"T" (single protected transaction), 
"X" (shares being swapped between two parties), 
"O" (ordinary private investors), 
"L" (late reported),
"M" (deal between two market makers),
"AT" (automated),
or no code (probably went to a different market such as PLUS).

\paragraph{Stop losses:}
It is crucial that traders show humility, are able to cut losses, accept mistakes and quickly move on.
Stop losses are designed for this purpose.
They are sell orders placed with the broker ordering them to sell your holding if the share price hits a predefined level.
They are preferable to mental stop losses as they can be placed in times of low stress, even when the markets are closed.
As a rule, stop losses should be set a minimum 10\% below the entry price.
This gap should be larger for volatile FTSE or small cap mining stocks due to market makers "treeshaking" the stock as means of triggering your stop loss.
Trailing stop-losses are offered by some brokers and track price rises by a minimum \%. 

\paragraph{Profit targets:}
Before opening a trade, decide at what price you are happy to sell the share.
The Naked Trader believes that profit targets should be at least 20\% higher than the buy price.
If a share is performing well, you may want to change the profit target to a higher level to maximise gains.

%------------------------------------------------
\section*{Section 5 - The next level}

\paragraph{SIPPs}
If you own frozen pension schemes or are yet to start a pension, consider opening a self invested personal pension (SIPP).
This allows you to invest your pension capital as if it were in your trading account.
You cannot withdraw the funds until you are 55 (at the time of writing).
If you are confident with your investment ability, this is a great way of taking charge of your own finances.

%------------------------------------------------
\section*{Links}
\begin{itemize}
\item http://uk.advfn.com/ - live share price and research.
\item http://www.investorschronicle.co.uk/ - UK equity research.
\item http://www.ibtimes.com/ - research.
\item www.allipo.com - check for IPOs.
\item http://www.proactiveinvestors.co.uk/ - Information on AIM listed companies and shareholder meetings/presentations.
\item https://www.amazon.co.uk/Investors-Guide-Charting-Analysis-Intelligent/dp/0273662031 - "The Investor's Guild to Charting" by Alistair Blair.
\end{itemize}

%------------------------------------------------------------------------------------------------